\documentclass[t]{beamer}
\usepackage{alltt}
\usepackage{verbatim}

\newcommand{\bs}{\symbol{`\\}}
\newcommand{\arr}{$\rightarrow$}
\newcommand{\LAMT}[1]{/\bs{}{#1} \arr }
\newcommand{\lam}[1]{\bs{}{#1} \arr }
\newcommand{\oomph}[1]{\textcolor[rgb]{0.0,0.5,0.9}{#1}}
\newcommand{\g}[1]{\textcolor[rgb]{0.7,0.3,0.3}{#1}}

\begin{document}
\title{SpecConstr: optimising purely functional loops}
\author{Amos Robinson}

\frame{\titlepage}

\begin{frame}[fragile]{Motivation - dot product}
The code we want to write
\begin{alltt}\Large
type V = Unboxed.Vector

dotp :: V Int -> V Int -> Int
dotp as bs
  = fold    (+) 0
  $ zipWith (*) as bs
\end{alltt}
\end{frame}

\begin{frame}[fragile]{Motivation - dot product}
The code we want to run
\begin{alltt}\Large
dotp as bs = go 0 0
 where
  go i acc
   | i > V.length as
   = acc
   | otherwise
   = go (i + 1) (acc + (as!i * bs!i))
\end{alltt}
No intermediate vectors, no constructors, no allocations: perfect.

(Just pretend they're not boxed ints...)
\end{frame}

\begin{frame}[fragile]{Motivation - dot product}
The code we get after stream fusion (trust me)
\begin{alltt}\Large
dotp as bs = go \oomph{(Nothing,} 0\oomph{)} 0
 where
  go (_, i) acc
   | i > V.length as
   = acc
  go (Nothing, i) acc
   = go (\oomph{Just} (as!i)\oomph{,} i\oomph{)} acc
  go (Just a, i) acc
   = go \oomph{(Nothing,} i + 1\oomph{)} (acc + (a * bs!i))
\end{alltt}
All those allocations!
\end{frame}

\begin{frame}[fragile]{Motivation - dot product}
The code we get after stream fusion (trust me)
\begin{alltt}\Large
dotp as bs = go (Nothing, 0) 0
 where
  go \oomph{(}_\oomph{,} i\oomph{)} acc
   | i > V.length as
   = acc
  go \oomph{(Nothing,} i\oomph{)} acc
   = go (Just (as!i), i) acc
  go \oomph{(Just} a\oomph{,} i\oomph{)} acc
   = go (Nothing, i + 1) (acc + (a * bs!i))
\end{alltt}
Only to be unboxed and scrutinised immediately. What a waste.
\end{frame}

\begin{frame}[fragile]{Motivation - dot product}
Let us try specialising this by hand.
\begin{alltt}
dotp as bs = \oomph{go (Nothing, 0) 0}
 where






  go (_, i) acc
   | i > V.length as  = acc
  go (Nothing, i) acc = go (Just (as!i), i) acc
  go (Just a,  i) acc = go (Nothing, i+1) (acc + (a*bs!i))
\end{alltt}
Start by looking at the first recursive call.
We can specialise the function for that particular call pattern.
\end{frame}

\begin{frame}[fragile]{Motivation - dot product}
Let us try specialising this by hand.
\begin{alltt}
dotp as bs = \oomph{go'1 0 0}
 where
  \oomph{go'1 i acc = case i > V.length as of
   True  -> acc
   False -> go (Just (as!i), i) acc}



  go (_, i) acc
   | i > V.length as  = acc
  go (Nothing, i) acc = go (Just (as!i), i) acc
  go (Just a,  i) acc = \oomph{go'1 (i + 1) (acc + (a * bs!i))}
\end{alltt}
Specialise on \verb/go (Nothing, x) y = go'1 x y/
\end{frame}

\begin{frame}[fragile]{Motivation - dot product}
Let us try specialising this by hand.
\begin{alltt}
dotp as bs = go'1 0 0
 where
  go'1 i acc = case i > V.length as of
   True  -> acc
   False -> \oomph{go (Just (as!i), i) acc}



  go (_, i) acc
   | i > V.length as  = acc
  go (Nothing, i) acc = go (Just (as!i), i) acc
  go (Just a,  i) acc = go'1 (i + 1) (acc + (a * bs!i))
\end{alltt}
Now look at the call in the new function. We can specialise on that pattern, too!
\end{frame}

\begin{frame}[fragile]{Motivation - dot product}
Let us try specialising this by hand.
\begin{alltt}
dotp as bs = go'1 0 0
 where
  go'1 i acc = case i > V.length as of
   True  -> acc
   False -> \oomph{go'2 (as!i) i acc}
  \oomph{go'2 a i acc = case i > V.length as of
   True  -> acc
   False -> go'1 (i + 1) (acc + (a * bs!i))}
  go (_, i) acc
   | i > V.length as  = acc
  go (Nothing, i) acc = go (Just (as!i), i) acc
  go (Just a,  i) acc = go'1 (i + 1) (acc + (a * bs!i))
\end{alltt}
Specialise on \verb/go (Just x, y) z = go'2 x y z/
\end{frame}

\begin{frame}[fragile]{Motivation - dot product}
Let us try specialising this by hand.
\begin{alltt}
dotp as bs = go'1 0 0
 where
  go'1 i acc = case i > V.length as of
   True  -> acc
   False -> go'2 (as!i) i acc
  go'2 a i acc = case i > V.length as of
   True  -> acc
   False -> go'1 (i + 1) (acc + (a * bs!i))
  \oomph{go (_, i) acc
   | i > V.length as  = acc
  go (Nothing, i) acc = go (Just (as!i), i) acc
  go (Just a,  i) acc = go'1 (i + 1) (acc + (a * bs!i))}
\end{alltt}
Now it turns out that \verb/go/ isn't even mentioned any more. Get rid of it.
\end{frame}

\begin{frame}[fragile]{Motivation - dot product}
Let us try specialising this by hand.
\begin{alltt}
dotp as bs = go'1 0 0
 where
  go'1 i acc = case i > V.length as of
   True  -> acc
   False -> \oomph{go'2 (as!i) i acc}
  go'2 a i acc = case i > V.length as of
   True  -> acc
   False -> \oomph{go'1 (i + 1) (acc + (a * bs!i))}




\end{alltt}
These two are mutually recursive, but we can still inline \verb/go'2/ into \verb/go'1/.
\end{frame}

\begin{frame}[fragile]{Motivation - dot product}
Let us try specialising this by hand.
\begin{alltt}
dotp as bs = go'1 0 0
 where
  go'1 i acc = \oomph{case i > V.length as of}
   True  -> acc
   \oomph{False -> case i > V.length as of}
      True  -> acc
      \oomph{False ->} go'1 (i + 1) (acc + (as!i * bs!i))





\end{alltt}
The case of \verb/i > V.length as/ is already inside the False branch of a case of the same expression,
we can remove the case altogether.
\end{frame}

\begin{frame}[fragile]{Motivation - dot product}
Let us try specialising this by hand.
\begin{alltt}
dotp as bs = go'1 0 0
 where
  go'1 i acc = case i > V.length as of
   True  -> acc
   False -> \oomph{go'1 (i + 1) (acc + (as!i * bs!i))}







\end{alltt}
Which was what we wanted.
\end{frame}





\begin{frame}[fragile]{GHC pipeline (not to scale)}
We now have some intuition about SpecConstr. How does it fit in with the rest of GHC's optimisations?

\begin{tabular}{lllll}
Parse      & :: & String & \arr & Source
\\
Typecheck  & :: & Source & \arr & Source
\\
Desugar     & :: & Source & \arr & Core
\\
Simplify    & :: & Core & \arr & Core
\\
SpecConstr & :: & Core & \arr & Core
\\
Simplify $\times$ 50 & :: & Core & \arr & Core
\\
Code generation & :: & Core & \arr & Object
\end{tabular}
\end{frame}

\begin{frame}[fragile]{GHC pipeline (not to scale)}
We now have some intuition about SpecConstr. How does it fit in with the rest of GHC's optimisations?

\begin{tabular}{lllll}
Parse     & :: & String & \arr & Source
\\
Typecheck  & :: & Source & \arr & Source
\\
Desugar     & :: & Source & \arr & Core
\\
\oomph{Simplify}    & :: & Core & \arr & Core
\\
\oomph{SpecConstr} & :: & Core & \arr & Core
\\
\oomph{Simplify $\times$ 50} & :: & Core & \arr & Core
\\
Code generation & :: & Core & \arr & Object
\end{tabular}

Focus on these parts.
\end{frame}


\begin{frame}[fragile]{Simplifier}
The simplifier does a bunch of transforms in a single pass:
\begin{itemize}
\item Case of constructor
\item Inlining
\item Rewrite rules
\item Let floating
\item Beta reduction
\end{itemize}
and many more, but these are the most interesting for us
\end{frame}

\begin{frame}[fragile]{Simplifier}
Case of constructor
\begin{alltt}\Large
case (\oomph{Just a}) of
 Nothing -> x
 \oomph{Just a'} -> \oomph{y}

==>

let \oomph{a'} = \oomph{a}
in  \oomph{y}
\end{alltt}
When the scrutinee of a case is known to be a constructor, we can remove the case altogether.
\end{frame}

\begin{frame}[fragile]{Simplifier}
Inlining
\begin{alltt}\Large
zipWith f xs ys
 = unstream $ zipWith_S f
   (stream xs) (stream ys)
...
zipWith (*) as bs

==>
...
unstream $ zipWith_S (*)
(stream as) (stream bs)
\end{alltt}
Move the definition of a function into places it is used
\end{frame}

\begin{frame}[fragile]{Simplifier}
Rewrite rules
\begin{alltt}\Large
\{-# RULES \oomph{stream (unstream xs)} = xs #-\}

fold_S (+) $ \oomph{stream $ unstream $}
zipWith_S (*) (stream as) (stream bs)
==>
...
fold_S (+) $
zipWith_S (*) (stream as) (stream bs)
\end{alltt}
Replace left-hand side with right, anywhere
\end{frame}

\begin{frame}[fragile]{SpecConstr, actually}
The basic idea:
\begin{itemize}
\item Find calls with constructors
\item Create new functions for that call pattern
\item Add rewrite rules for each call pattern
\item Let the simplifier do the rest
\end{itemize}

\begin{alltt}
enumFromTo f t acc
 = case f > t of
   True  -> acc
   False -> \oomph{enumFromTo f (t-1) (t : acc)}
\end{alltt}

(Silly example.)

\end{frame}

\begin{frame}[fragile]{SpecConstr, actually}
The basic idea:
\begin{itemize}
\item Find calls with constructors
\item Create new functions for that call pattern
\item Add rewrite rules for each call pattern
\item Let the simplifier do the rest
\end{itemize}

\begin{alltt}
enumFromTo f t acc
 = case f > t of
   True  -> acc
   False -> enumFromTo'1 f (t-1) t acc
\oomph{enumFromTo'1 f t cons acc
 = case f > t of
   True  -> acc
   False -> enumFromTo f (t-1) (t : cons : acc)}
\end{alltt}
Not only will this diverge, it's not even decreasing allocations!
\end{frame}

\begin{frame}[fragile]{SpecConstr, actually}
The basic idea:
\begin{itemize}
\item Find calls with constructors \oomph{on scrutinised parameters}
\item Create new functions for that call pattern
\item Add rewrite rules for each call pattern
\item Let the simplifier do the rest
\end{itemize}

\begin{alltt}
enumFromTo f t acc
 = case f > t of
   True  -> acc
   False -> enumFromTo f (t-1) (t : acc)
\end{alltt}

\end{frame}

\begin{frame}[fragile]{SpecConstr, actually}
Looking through bindings

\begin{alltt}
silly2 xs' = case xs' of
 []     -> []
 (x:xs) -> if   x > 10
           then (do1 \oomph{(x:xs)}, do2 \oomph{(x:xs)}) : silly2 \oomph{(x:xs)}
           else silly2 xs
\end{alltt}

Common subexpression elimination (CSE) will probably rewrite those \verb/x:xs/ into \verb/xs'/.
\end{frame}

\begin{frame}[fragile]{SpecConstr, actually}
Looking through bindings

\begin{alltt}
silly2 xs' = case xs' of
 []     -> []
 (x:xs) -> if   x > 10
           then (do1 \oomph{   xs'}, do2 \oomph{   xs'}) : silly2 \oomph{   xs'}
           else silly2 xs
\end{alltt}

But now it's not obvious that \verb/silly2 xs'/ is a valid call pattern.
No matter: keep track of the bound variables and their values. If we know \verb/xs'=x:xs/, we can still specialise.
\end{frame}

\begin{frame}[fragile]{SpecConstr, actually}
Reboxing

\begin{alltt}
silly2 xs' = case xs' of
 []     -> []
 (x:xs) -> if   x > 10
           then (do1    xs', do2    xs') : \oomph{silly2    xs'}
           else silly2 xs
\end{alltt}

Now we'll specialise on \verb/silly2 (x:xs) = silly2'1 x xs/.
\end{frame}

\begin{frame}[fragile]{SpecConstr, actually}
Reboxing

\begin{alltt}
silly2 xs' = case xs' of
 []     -> []
 (x:xs) -> if   x > 10
           then (do1    xs', do2    xs') : silly2    xs'
           else silly2 xs
silly2'1 x xs
 =         if   x > 10
           then (do1 \oomph{(x:xs)}, do2 \oomph{(x:xs)}) : \oomph{silly2'1 x xs}
           else silly2 xs
\end{alltt}

Hey! Now we're actually doing \emph{more} allocations.

The moral: don't specialise on a bound variable if the variable is used elsewhere. 
\end{frame}


\begin{frame}[fragile]{ForceSpecConstr}
SpecConstr puts a limit on the number of specialisations,
to prevent code blowup.
\begin{alltt}
unstream :: Stream a -> [a]
unstream (Stream f s) = go \oomph{ForceSpecConstr} s
 where
  go \oomph{ForceSpecConstr} s
   = case f s of
     Done       -> []
     Skip    s' ->     go \oomph{ForceSpecConstr} s'
     Yield a s' -> a : go \oomph{ForceSpecConstr} s'
\end{alltt}

But with stream fusion, we want to specialise everything no matter what.
Damn the consequences!
\end{frame}




\begin{frame}[fragile]{End}
end.
\end{frame}



\end{document}
