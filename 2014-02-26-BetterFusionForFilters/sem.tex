\documentclass{beamer}
\usepackage{alltt}
\usepackage{verbatim}
\usepackage{graphicx}

\usepackage{tikz}
\usepackage{svg}

\newcommand{\bs}{\symbol{`\\}}
\newcommand{\arr}{$\rightarrow$}
\newcommand{\LAMT}[1]{/\bs{}{#1} \arr }
\newcommand{\lam}[1]{\bs{}{#1} \arr }
\newcommand{\bl}[1]{\textcolor[rgb]{0.0,0.5,0.9}{#1}}
\newcommand{\g}[1]{\textcolor[rgb]{0.7,0.3,0.3}{#1}}
\newcommand{\off}[1]{\textcolor[rgb]{0.7,0.7,0.7}{#1}}

\newcommand{\fr}[1]{\begin{frame}[fragile]{#1}}

\begin{document}
\title{Better fusion for filters}
\author{Amos Robinson\\PhD student at UNSW}

\frame{\titlepage}


\fr{I want my programs to be fast}
But I don't want to have to write this kind of rubbish.

\begin{alltt}
int  *\bl{out}    = malloc(\bl{in_len});
int   \bl{out_ix} = 0;
int   \bl{fold_m} = 0;
int   \bl{in_ix}  = 0;

\g{for} (; \bl{in_ix} != \bl{in_len}; ++\bl{in_ix}) \{
  int \bl{elm} = \bl{in}[\bl{in_ix}] + 1;
  \bl{fold_m}  = max(\bl{fold_m}, \bl{elm});
  \g{if} (\bl{elm} > 0) \{
      \bl{out}[\bl{out_ix}++] = \bl{elm};
  \}
\}

*\bl{out_ptr} = \bl{out};
*\bl{out_len} = \bl{out_ix};
*\bl{fold_res}= \bl{fold_m};
\end{alltt}

\end{frame}



\fr{Haskell is alright for expressing high-level programs}
But the high-level version isn't necessarily the fastest.

\begin{alltt}
filterMax (\bl{vs} : Vector Int) =
 let \bl{vs'} = \g{map}    (+1)    \bl{vs}
     \bl{m}   = \g{fold}     0 max \bl{vs'}
     \bl{vs''}= \g{filter} (>0)    \bl{vs'}
 in (\bl{m}, \bl{vs''})
\end{alltt}

\end{frame}


\fr{How many loops in this program?}
About 3.

\begin{alltt}
filterMax (\bl{vs} : Vector Int) =
 let \bl{vs'} = \g{map}    (+1)    \bl{vs}
     \bl{m}   = \g{fold}     0 max \bl{vs'}
     \bl{vs''}= \g{filter} (>0)    \bl{vs'}
 in (\bl{m}, \bl{vs''})
\end{alltt}

\end{frame}


\fr{So, we need to merge the loops together}
\begin{itemize}
\item
This is called \emph{fusion}
\item
But we must be careful not to modify the program's behaviour
\end{itemize}
\end{frame}

\fr{There are restrictions}
Only loops of the same size can be fused
\begin{alltt}
different (\bl{as} \bl{bs} : Vector Int) =
 let \bl{as'} = \g{map} (+1) \bl{as}
     \bl{bs'} = \g{map} (-1) \bl{bs}
 in (\bl{as'}, \bl{bs'})
\end{alltt}
No fusion here: \bl{\tt as} and \bl{\tt bs} may be different lengths.
\end{frame}

\fr{A fold can't be fused with its output}
Because a fold consumes the entire input before producing anything.
\begin{alltt}
normalise (\bl{as} : Vector Int) =
 let \bl{sum} = \g{fold} (+) 0  \bl{as}
     \bl{as'} = \g{map}  (/\bl{sum}) \bl{as}
 in  \bl{as'}
\end{alltt}
Two loops: \bl{\tt sum}; \bl{\tt as'}
\end{frame}

\fr{What of filters?}
Can we fuse operations on filtered data with the original?

\begin{alltt}
sum2       (\bl{as} : Vector Int) =
 let \bl{filt}  = \g{filter} (>0)  \bl{as}
     \bl{s1}    = \g{fold}   (+) 0 \bl{filt}
     \bl{s2}    = \g{fold}   (+) 0 \bl{as}


 in (\bl{s1}, \bl{s2})
\end{alltt}
One loop: \bl{\tt filt s1 s2}
\end{frame}


\fr{What of filters?}
% \begin{itemize}
% \item
% The output of a filter isn't the same size as its input
% \item
% But the \emph{loop} that produces the filtered result is the same size
% \item
% So the loop size of some operation on filtered data depends on the results of fusion!
% \end{itemize}
Only if they're in the generator loop

\begin{alltt}
normalise2 (\bl{as} : Vector Int) =
 let \bl{filt}  = \g{filter} (>0)  \bl{as}
     \bl{s1}    = \g{fold}   (+) 0 \bl{filt}
     \bl{s2}    = \g{fold}   (+) 0 \bl{as}
     \bl{filt'} = \g{map}    (/\bl{s1}) \bl{filt}
     \bl{as'}   = \g{map}    (/\bl{s2}) \bl{as}
 in (\bl{filt'}, \bl{as'})
\end{alltt}
Three loops: \bl{\tt filt s1 s2}; \bl{\tt filt'}; \bl{\tt as'}
\end{frame}



\fr{There may be multiple ways of fusing a program}
How do we choose which one?
\begin{alltt}
multiple (\bl{us} : Vector Int) =
 let \bl{xs} = \g{map}  (+1)  \bl{us}
     \bl{y}  = \g{fold} (+) 0 \bl{us}
     \bl{ys} = \g{map}  (+y)  \bl{xs}
 in  \bl{ys}
\end{alltt}
\end{frame}


\fr{Choice 1}
3 loops, 2 manifest arrays
\begin{alltt}
multiple (\bl{us} : Vector Int) =
 let \bl{xs}      = \g{loop #1: map  xs}
     \bl{y}       = \g{loop #2: fold y}
     \bl{ys}      = \g{loop #3: map  ys}
 in  \bl{ys}
\end{alltt}
\end{frame}


\fr{Choice 2}
2 loops, 2 manifest arrays
\begin{alltt}
multiple (\bl{us} : Vector Int) =
 let \bl{(xs, y)} = \g{loop #1: map  xs, fold y}
     
     \bl{ys}      = \g{loop #2: map  ys}
 in  \bl{ys}
\end{alltt}
\end{frame}


\fr{Choice 3 - ding!}
2 loops, 1 manifest array
\begin{alltt}
multiple (\bl{us} : Vector Int) =
 let \bl{y}       = \g{loop #1: fold y}
     
     \bl{ys}      = \g{loop #2: map  xs, map  ys}
 in  \bl{ys}
\end{alltt}
\end{frame}


\fr{My rough definition of `best' is}
Minimise manifest arrays, then minimise number of loops.
\end{frame}


\fr{How do we do it?}
Let us try on a simple program that has filters
\begin{alltt}
filt2 \bl{ins} =
 let \bl{us} = \g{filter} p \bl{ins}
     \bl{vs} = \g{filter} q \bl{ins}
     \bl{ys} = \g{map}    f \bl{us}
     \bl{zs} = \g{map}    g \bl{vs}
 in (\bl{ys}, \bl{zs})
\end{alltt}
\end{frame}


\fr{Size inference}
Find out what we can about sizes of arrays
\begin{alltt}
filt2 \bl{ins} =            -- \bl{ins} : Vector \bl{k}  Int
 let \bl{us} = \g{filter} p \bl{ins} -- \bl{us}  : Vector \bl{?u} Int, \bl{?u} <= \bl{k}
     \bl{vs} = \g{filter} q \bl{ins} -- \bl{vs}  : Vector \bl{?v} Int, \bl{?v} <= \bl{k}
     \bl{ys} = \g{map}    f \bl{us}  -- \bl{ys}  : Vector \bl{?u} Int
     \bl{zs} = \g{map}    g \bl{vs}  -- \bl{zs}  : Vector \bl{?v} Int
 in (\bl{ys}, \bl{zs})
\end{alltt}
\end{frame}


\fr{Integer linear programming}
Find a variable assignment that minimises the goal, satisfying constraints.
\begin{alltt}
Minimise   2x +  y
Subject to  x +  y   >= 2
            x + 2y   >= 4

==>

x = 0,
y = 2
\end{alltt}
\end{frame}


\fr{A variable for each pair of nodes}
For each pair of distinct nodes \bl{i} and \bl{j}, we have:
\begin{itemize}
\item
boolean variable \bl{\tt C(i,j)}, 0 if fused together
\item
constant weight \bl{\tt W(i,j)}, the benefit of fusion
% - say, 100 if fusing \bl{i} and \bl{j} would remove a manifest array, 1 otherwise
\end{itemize}

\begin{alltt}
Minimise   Sum \bl{W(i,j)} * \bl{C(i,j)}
Subject to ...
\end{alltt}
\end{frame}

\fr{Simple constraints}
Not all nodes can be fused together, so we must add some constraints for this:
\begin{itemize}
\item
\g{\tt fold}s cannot be fused with their outputs
\item
nodes of completely different sizes cannot be fused
\end{itemize}

\begin{alltt}
Minimise   Sum \bl{W(i,j)} * \bl{C(i,j)}
Subject to
           \bl{C(i,j)} = 1 if \bl{i} is a \g{fold}
           \bl{C(i,j)} = 1 if \bl{i} and \bl{j} are different sizes
\end{alltt}
\end{frame}


\fr{Filter constraints}
Operations on filtered data, despite having different sizes, \emph{may} sometimes be fused:

\begin{alltt}
filt2 \bl{ins} =
 let \bl{us} = \g{filter} p \bl{ins}
     \bl{vs} = \g{filter} q \bl{ins}
     \bl{ys} = \g{map}    f \bl{us}
     \bl{zs} = \g{map}    g \bl{vs}
 in (\bl{ys}, \bl{zs})

           \bl{C(us,ys)} = 1 ==> \bl{C(ys,zs)} = 1
           \bl{C(vs,zs)} = 1 ==> \bl{C(ys,zs)} = 1
           \bl{C(us,vs)} = 1 ==> \bl{C(ys,zs)} = 1
\end{alltt}
\end{frame}


\fr{Acyclic constraint}
The last thing we need is to ensure the clustering is acyclic; otherwise we'd need to execute two clusters at once.
\begin{itemize}
\item
Give each node a number \bl{\tt pi(i)}
\item
If \bl{\tt C(i,j) = 0} then \bl{\tt pi(i) = pi(j)}
\item
If j is after i, \bl{\tt pi(i) < pi(j)}
\end{itemize}

\begin{alltt}
           \bl{C(i,j)} <= \bl{pi(j)} - \bl{pi(i)} <= 100*\bl{C(i,j)}
\end{alltt}
\end{frame}


\fr{Put it all together}

\begin{alltt}
Minimise   Sum \bl{W(i,j)} * \bl{C(i,j)}
Subject to 
           \bl{C(us,ys)} = 1 ==> \bl{C(ys,zs)} = 1
           \bl{C(vs,zs)} = 1 ==> \bl{C(ys,zs)} = 1
           \bl{C(us,vs)} = 1 ==> \bl{C(ys,zs)} = 1

           \bl{C(us,ys)} <= \bl{pi(ys)} - \bl{pi(us)} <= 100*\bl{C(us,ys)}
           \bl{C(vs,zs)} <= \bl{pi(zs)} - \bl{pi(vs)} <= 100*\bl{C(vs,zs)}
\end{alltt}
\end{frame}


\end{document}
